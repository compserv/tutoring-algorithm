% Thanks to Dan Wang for the Beamer template!

\documentclass[xcolor=svgnames, 18pt]{beamer}

% Fill in variables below %
% ++++++++++++++++++++++++++++++++++++++++++++++++++++++++++++++++++++++++++++ %
 \newcommand{\thesemester}{Spring 2013}
\newcommand{\themidterm}{Improvements \\ on a \\ Time Slot Allocation Algorithm}
\newcommand{\theauthors}{Lewin Gan, Allen Xiao, Edwin Liao}
\newcommand{\theorganization}{CS270, Spring 2013 \\
University of California, Berkeley}
\newcommand{\thedate}{May 9, 2013}
\newcommand{\thelanguage}{Java}
% ++++++++++++++++++++++++++++++++++++++++++++++++++++++++++++++++++++++++++++ %

% Preamble %
% ---------------------------------------------------------------------------- %
\usepackage{url}
\usepackage{relsize}
\usepackage{color}
\usepackage{listings}
\usepackage{multirow}
\usepackage{array}
\usepackage{bm}
\usepackage{framed}
\usepackage{amsfonts}

% Listings Package %
\usepackage{listings}
\lstset{numbers=left,
    basicstyle=\ttfamily,
    numberstyle=\small,
    showstringspaces=false,
    frame=leftline,
    language=Java,
    escapeinside=\$\$,
    keywordstyle=\color{blue},
    xleftmargin=16pt,
    morecomment=[l]{//},
    }

\usetheme{Singapore}
\usecolortheme[named=DarkSlateBlue]{structure} 
\setbeamertemplate{mini frames}{}
\beamertemplatenavigationsymbolsempty
\setbeamertemplate{footline}[frame number]


\title{\themidterm}
\author{\theauthors}
\institute{\theorganization}
\date{\thedate}
% ---------------------------------------------------------------------------- %

\begin{document}

% Title Page %
% ---------------------------------------------------------------------------- %

\begin{frame}[fragile]
  \titlepage
\end{frame}


% Slide 1: Motivation %
% ---------------------------------------------------------------------------- %

\begin{frame}[fragile]
  \frametitle{Motivation} 

\uncover<2->{
\begin{itemize}
\item Schedule tutors into time/location slots
\begin{itemize}
\item HKN's undergrad tutoring hours
\end{itemize}
}

\uncover<3->{
\item Old implementation:
\begin{itemize}
\item Linear programming to find a feasible matching
\item Unoptimized hill-climbing to improve solution
\item Ran overnight to get satisfactory solution
\end{itemize}
}

\uncover<4->{
\item \emph{General Assignment Problem} (GAP)
\begin{itemize}
\item $n$ agents and $m$ tasks
\item Each agent has a max capacity for tasks
\item Minimize the total cost of assigning all tasks
\end{itemize}
}

\uncover<5->{
\item Maximum General Assignment Problem
\begin{itemize}
\item Still misses some parts of the model
\end{itemize}
}

\end{itemize}
\end{frame}


% Slide 2: Features %
% ---------------------------------------------------------------------------- %

\begin{frame}[fragile]
  \frametitle{Features} 

\uncover<2->{
\begin{itemize}
\item Slots have preferences on:
\begin{itemize}
\item Courses tutored at that slot's location
\end{itemize}
}

\uncover<3->{
\item Tutors have preferences on:
\begin{itemize}
\item Courses they are confident in
\item Times, locations
\item Whether their assigned slots are adjacent
\end{itemize}
}

\uncover<4->{
\item Hard to encode the last preference without making the problem exponential in size
}

\end{itemize}
\end{frame}


% Slide 3: A New Approach %
% ---------------------------------------------------------------------------- %

\begin{frame}[fragile]
  \frametitle{A New Approach}

\uncover<2->{
\begin{itemize}
\item Assume $|S| \le k|T|$, where $k$ is the number of slots a tutor needs to take
}

\uncover<3->{
\item Then each tutor will have exactly $k$ slots, and each slot has at least one tutor
}

\uncover<4->{
\item Then, while we have remaining slots:
}

\begin{itemize}
\uncover<5->{
\item Use maximum weight matching to find an assignment from tutors to slots remaining
}

\uncover<6->{
\item Remove the slots that were matched ($|T|$ of them)
}

\end{itemize}

\uncover<7->{
\item This will run at most $k$ times
}

\end{itemize}
\end{frame}


% Slide 4: Why this Approach? %
% ---------------------------------------------------------------------------- %

\begin{frame}[fragile]
  \frametitle{Why this Approach?}

\uncover<2->{
\begin{itemize}
\item Weights change depending on what is already matched

\begin{itemize}
\item (i.e. Tutor prefers slots that are adjacent)
\end{itemize}
}

\uncover<3->{
\item Some matchings infeasible

\begin{itemize}
\item (Tutors can not go to two slots that are at the same time)
\end{itemize}
}

\uncover<4->{
\item Iterations allow us to change weights in between
}

\end{itemize}
\end{frame}


% Slide 5: How Good is this Approach? %
% ---------------------------------------------------------------------------- %

\begin{frame}[fragile]
  \frametitle{How Good is this Approach?}

\uncover<2->{
\begin{itemize}
\item At each iteration, we choose a subset of slots such that the weights matching with tutors is optimal

\begin{itemize}
\item Let the first iteration have value $X$
\end{itemize}
}

\uncover<3->{
\item Then, any subset of $|T|$ slots maximally matched with the tutors will have value $\le X$
}

\uncover<4->{
\item Worst case: algorithm will only find one such subset $X$, and other subsets zero; and optimal could have all subsets $X$

\begin{itemize}
\item Total value of $kX$
\end{itemize}
}

\uncover<5->{
\item Thus, we have a $k$-approximation (in our case, $k=2$)
}

\end{itemize}
\end{frame}


\end{document}
